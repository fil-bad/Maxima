%& -shell-escape
\documentclass[a4paper,12pt]{article}
\usepackage[legalpaper, portrait , margin=1.5cm]{geometry}
\usepackage[latin1]{inputenc}
\usepackage[T1]{fontenc}
\usepackage[italian]{babel}
\usepackage{wrapfig}
\usepackage{graphicx}
\graphicspath{ {./Sorgenti/} }
\usepackage{enumitem}
\usepackage{pifont}
\usepackage{amsmath}

\author{Alfano Emanuele \\ Badalamenti Filippo \\ Vitti Gabriele}

\begin{document}
\title{Cinematiche inverse Disaccoppiate}
\maketitle

In questa relazione andremo a vedere come si calcolano le cinematiche
  inverse totali di 2 robot:

\begin{itemize}
\item Stanford + Polso Sferico
\item Antropomorfo + Polso Sferico
\end{itemize}

Per prima cosa è necessario trovare i 2 termini costanti che permettono la
disaccoppiazione della struttura con il polso
\pagebreak

\section{Calcolo costanti Polso Sferico}
Data la matrice totale del polso sferico:

\begin{center}
$Q_{46}$ =$\begin{pmatrix}
{c_{{q_4}}} {c_{{q_5}}} {c_{{q_6}}}-{s_{{q_4}}} {s_{{q_6}}} & -{c_{{q_4}}} {c_{{q_5}}} {s_{{q_6}}}-{s_{{q_4}}} {c_{{q_6}}} & {c_{{q_4}}} {s_{{q_5}}} & {L_6} {c_{{q_4}}} {s_{{q_5}}}\\
{c_{{q_4}}} {s_{{q_6}}}+{s_{{q_4}}} {c_{{q_5}}} {c_{{q_6}}} & {c_{{q_4}}} {c_{{q_6}}}-{s_{{q_4}}} {c_{{q_5}}} {s_{{q_6}}} & {s_{{q_4}}} {s_{{q_5}}} & {L_6} {s_{{q_4}}} {s_{{q_5}}}\\
-{s_{{q_5}}} {c_{{q_6}}} & {s_{{q_5}}} {s_{{q_6}}} & {c_{{q_5}}} & {L_6} {c_{{q_5}}}\\
0 & 0 & 0 & 1
\end{pmatrix}$
\end{center}
Abbiamo che:

\begin{center}
$R_{46}$ = $\begin{pmatrix}{c_{{q_4}}} {c_{{q_5}}} {c_{{q_6}}}-{s_{{q_4}}} {s_{{q_6}}} & -{c_{{q_4}}} {c_{{q_5}}} {s_{{q_6}}}-{s_{{q_4}}} {c_{{q_6}}} & {c_{{q_4}}} {s_{{q_5}}}\\
{c_{{q_4}}} {s_{{q_6}}}+{s_{{q_4}}} {c_{{q_5}}} {c_{{q_6}}} & {c_{{q_4}}} {c_{{q_6}}}-{s_{{q_4}}} {c_{{q_5}}} {s_{{q_6}}} & {s_{{q_4}}} {s_{{q_5}}}\\
-{s_{{q_5}}} {c_{{q_6}}} & {s_{{q_5}}} {s_{{q_6}}} & {c_{{q_5}}}\end{pmatrix}$
\space\space\space\space\space   
$d_{46}$ = $\begin{pmatrix}{L_6} {c_{{q_4}}} {s_{{q_5}}}\\
{L_6} {s_{{q_4}}} {s_{{q_5}}}\\
{L_6} {c_{{q_5}}}\end{pmatrix}$
\end{center}

Da queste 2 matrici cerchiamo di trovare le 2 costanti che rendono vera la seguente equazione:

$$ d_{36} = R_{36} \cdot d_1 + d_0 $$

Trovando queste 2 costanti sarà infatti possibile disaccoppiare la soluzione delle 2 cinematiche inverse.





\begin{center}
$\begin{pmatrix}{L_6} {c_{{q_4}}} {s_{{q_5}}}\\
{L_6} {s_{{q_4}}} {s_{{q_5}}}\\
{L_6} {c_{{q_5}}}\end{pmatrix}$
=
$\begin{pmatrix}{c_{{q_4}}} {c_{{q_5}}} {c_{{q_6}}}-{s_{{q_4}}} {s_{{q_6}}} & -{c_{{q_4}}} {c_{{q_5}}} {s_{{q_6}}}-{s_{{q_4}}} {c_{{q_6}}} & {c_{{q_4}}} {s_{{q_5}}}\\
{c_{{q_4}}} {s_{{q_6}}}+{s_{{q_4}}} {c_{{q_5}}} {c_{{q_6}}} & {c_{{q_4}}} {c_{{q_6}}}-{s_{{q_4}}} {c_{{q_5}}} {s_{{q_6}}} & {s_{{q_4}}} {s_{{q_5}}}\\
-{s_{{q_5}}} {c_{{q_6}}} & {s_{{q_5}}} {s_{{q_6}}} & {c_{{q_5}}}\end{pmatrix}$
$\cdot$
$\begin{pmatrix}\mathit{a1}\\
\mathit{b1}\\
\mathit{c1}\end{pmatrix}$
+
$\begin{pmatrix}\mathit{a0}\\
\mathit{b0}\\
\mathit{c0}\end{pmatrix}$
\end{center}

Dove:
\begin{center}
$ d_1 $=$\begin{pmatrix}\mathit{a1}\\
\mathit{b1}\\
\mathit{c1}\end{pmatrix}$
\space
$ d_0 $=$\begin{pmatrix}\mathit{a0}\\
\mathit{b0}\\
\mathit{c0}\end{pmatrix}$
\end{center}


Risolvendo sia con $d_1$ che con $d_2$ si ottengono $\infty$ sol, ma si vede anche a occhio che una soluzione semplice è:

\begin{center}
$ d_1 $=$\begin{pmatrix}
0\\
0\\
L_6\end{pmatrix}$
\space
$ d_0 $=$\begin{pmatrix}
0\\
0\\
0\end{pmatrix}$
\end{center}

\section{Orientamento inverso Polso Sferico}
Per calcolare l'orientamento inverso del polso sferico non ci serve sapere null'altro all'infuori dell'orientamento stesso.
Con le costanti potremo calcolare quindi l'obiettivo, ma le forumule sono note a prescindere, calcoliamo quindi subito l'orientamento inverso cosi  da poter riprendere i calcoli dopo.


Nel calcolo disaccoppiato avremo che la matrice del polso sferico dovrà essere pari a:

\begin{center}
$ R_{03}^T(q_a) \cdot R = R_{36}(q_b) $
 \end{center}


Essedo la matrice di orientamento del polso sferico($R_{36}$) esattamente
pari alla matrice di orientamento della terna $ R_{zyz} $ nei 5 campi semplici,
trovare $ \alpha,\beta,\gamma $ della matrice $ R_{zyz}(\alpha,\beta,\gamma) $ equivale a trovare $q_4,q_5,q_6$ 

\begin{center}
$ R_{zyz}(\alpha,\beta,\gamma)$ = $\begin{pmatrix}\operatorname{c}\left( \alpha \right)  \operatorname{c}\left( \beta \right)  \operatorname{c}\left( \gamma \right) -\operatorname{s}\left( \alpha \right)  \operatorname{s}\left( \gamma \right)  & -\operatorname{c}\left( \alpha \right)  \operatorname{s}\left( \gamma \right) -\operatorname{s}\left( \alpha \right)  \operatorname{c}\left( \beta \right)  \operatorname{c}\left( \gamma \right)  & \operatorname{s}\left( \beta \right)  \operatorname{c}\left( \gamma \right) \\
\operatorname{c}\left( \alpha \right)  \operatorname{c}\left( \beta \right)  \operatorname{s}\left( \gamma \right) +\operatorname{s}\left( \alpha \right)  \operatorname{c}\left( \gamma \right)  & \operatorname{c}\left( \alpha \right)  \operatorname{c}\left( \gamma \right) -\operatorname{s}\left( \alpha \right)  \operatorname{c}\left( \beta \right)  \operatorname{s}\left( \gamma \right)  & \operatorname{s}\left( \beta \right)  \operatorname{s}\left( \gamma \right) \\
-\operatorname{c}\left( \alpha \right)  \operatorname{s}\left( \beta \right)  & \operatorname{s}\left( \alpha \right)  \operatorname{s}\left( \beta \right)  & \operatorname{c}\left( \beta \right) \end{pmatrix}$
\end{center}


Definito: $R_{03}^T(q_a) \cdot R$ = $\begin{pmatrix}{r_{11}} & {r_{12}} & {r_{13}}\\
{r_{21}} & {r_{22}} & {r_{23}}\\
{r_{31}} & {r_{32}} & {r_{33}}\end{pmatrix}$.
Avremo che $q_4,q_5,q_6$ sono:

\begin{itemize}
\item $c_{q_5} = r_{33} \Longrightarrow s_{q_5} = \pm \sqrt{1-r_{33}^2}
		\hfill \longrightarrow \hfill
		q_5 = atan2 (\pm \sqrt{1-r_{33}^2},r_{33}) = \left \{ \begin{array}{rl}
		q_5\\
		-q_5
		\end{array}
		\right.$
\end{itemize}
Se $q_5 \neq 0$
\begin{itemize}	

\item $\left \{ \begin{array}{rl}
		s_{q_4}s_{q_5}=r_{32} \\
		-c_{q_4}s_{q_5}=r_{31}
		\end{array}
		\right. %il punto serve per non far chiamare errore
		\longrightarrow
		\left \{ \begin{array}{rl}
		s_{q_4}=\dfrac{r_{32}}{s_{q_5}} \\
		-c_{q_4}=\dfrac{r_{31}}{s_{q_5}}
		\end{array}
		\right.
		\longrightarrow
		\left \{ \begin{array}{rl}
		s_{q_4}=\pm r_{32}\\
		c_{q_4}=\mp r_{31}
		\end{array}
		\right.
		\longrightarrow $
				
		$q_4 = atan2 (\pm r_{32},\mp r_{31}) =
		\left \{ \begin{array}{rl}
		q_4\\
		q_4+\pi
		\end{array}
		\right.$
		
		
\item $\left \{ \begin{array}{rl}
		s_{q_5}c_{q_6}=r_{13} \\
		s_{q_5}s_{q_6}=r_{23}
		\end{array}
		\right. %il punto serve per non far chiamare errore
		\longrightarrow
		\left \{ \begin{array}{rl}
		s_{q_6}=\dfrac{r_{13}}{s_{q_5}} \\
		c_{q_6}=\dfrac{r_{23}}{s_{q_5}}
		\end{array}
		\right.
		\longrightarrow
		\left \{ \begin{array}{rl}
		s_{q_4}=\pm r_{13}\\
		-c_{q_4}=\pm r_{23}
		\end{array}
		\right.
		\longrightarrow $
				
		$q_6 = atan2 (\pm r_{13},\mp r_{23}) =
		\left \{ \begin{array}{rl}
		q_6\\
		q_6+\pi
		\end{array}
		\right.$		
\end{itemize}
\newpage
\section{Stanford + Polso Sferico}
\subsection{Calcolo posizione disaccoppiata}

Avendo noi già calcolato le 2 costanti di disaccoppiamento possiamo calcolare subito che:

Dati: $d_f=\begin{pmatrix}{{\mathit{df}}_1}\\
{{\mathit{df}}_2}\\
{{\mathit{df}}_3}\end{pmatrix}$ e $R_f=\begin{pmatrix}{r_{11}} & {r_{12}} & {r_{13}}\\
{r_{21}} & {r_{22}} & {r_{23}}\\
{r_{31}} & {r_{32}} & {r_{33}}\end{pmatrix}$


L'equazione per ottenere le coordinate obiettivo della struttura e':
\begin{center}
$d_f- R_f \cdot d_1 = R_{03}(q_a) \cdot d_0 + d_{03}(q_a)$ 

Ovvero sia, svolgendo il calcolo:

$\begin{pmatrix}{{\mathit{df}}_1}-{L_6} {r_{13}}\\
{{\mathit{df}}_2}-{L_6} {r_{23}}\\
{{\mathit{df}}_3}-{L_6} {r_{33}}\end{pmatrix}=\begin{pmatrix}-{L_2} {s_{{q_1}}}+{c_{{q_1}}} {s_{{q_2}}} {q_3}\\
{L_2} {c_{{q_1}}}+{s_{{q_1}}} {s_{{q_2}}} {q_3}\\
{L_1}+{c_{{q_2}}} {q_3}\end{pmatrix}$
\end{center}

Applicando la cinematica inversa dello stanford viene quindi fuori:
\begin{itemize}
%\item $q_3 = \sqrt{(df_1-L_6 r_{13})^2 +(df_2-L_6 r_{23})^2+(df_3-L_6 r_{33}-L_1)-L_2^2}$
\item $q_3 = \sqrt{x^2 +y^2+(z-L_1)-L_2^2}$
\item $q_2 = atan2(\pm\sqrt{\dfrac{x^2+y^2-L2^2}{q_3^2}},\dfrac{z-L_1}{q_3})$
\item $q_1 = atan2(q_3 s_{q_2}y-L_2x,q_3s_{q2}x+L_2y)$
\end{itemize}

\subsection{Calcolo Orientamento disaccoppiato}
Arrivati qui basta calcolare il risultato di $ R_{03}^T(q_a) \cdot R $ per ottenere la matrice da eguagliare a $R_{36}(q_b)$
Calcolato il risultato e' possibile risalire quindi anche ai valori degli ultimi 3 giunti

\newpage

\section{Puma (Antropomorfo + Polso Sferico)}
\subsection{Calcolo posizione disaccoppiata}
Come prima procediamo calcolando la posizione che il braccio dovra' avere dati:
\begin{center}
 $d_f=\begin{pmatrix}{{\mathit{df}}_1}\\
{{\mathit{df}}_2}\\
{{\mathit{df}}_3}\end{pmatrix}$ e $R_f=\begin{pmatrix}{r_{11}} & {r_{12}} & {r_{13}}\\
{r_{21}} & {r_{22}} & {r_{23}}\\
{r_{31}} & {r_{32}} & {r_{33}}\end{pmatrix}$
\end{center}

L'equazione per ottenere le coordinate obiettivo della struttura e':
\begin{center}
$d_f- R_f \cdot d_1 = R_{03}(q_a) \cdot d_0 + d_{03}(q_a)$ 

Ovvero sia, svolgendo il calcolo:

$\begin{pmatrix}{{\mathit{df}}_1}-{L_6} {r_{13}}\\
{{\mathit{df}}_2}-{L_6} {r_{23}}\\
{{\mathit{df}}_3}-{L_6} {r_{33}}\end{pmatrix}=\begin{pmatrix}{D_2} {c_{{q_1}}} {c_{{q_2}}}+{D_3} {c_{{q_1}}} {c_{{q_2}}} {c_{{q_3}}}+-{D_3} {c_{{q_1}}} {s_{{q_2}}} {s_{{q_3}}}\\
{D_2} {s_{{q_1}}} {c_{{q_2}}}+{D_3} {s_{{q_1}}} {c_{{q_2}}} {c_{{q_3}}}+-{D_3} {s_{{q_1}}} {s_{{q_2}}} {s_{{q_3}}}\\
{L_1}+{D_2} {s_{{q_2}}}+{D_3} {s_{{q_2}}} {c_{{q_3}}}+{D_3} {c_{{q_2}}} {s_{{q_3}}}\end{pmatrix}$
\end{center}

Applicando la cinematica inversa dell'antropomorfo viene quindi fuori:
\begin{itemize}
%\item $q_3 = \sqrt{(df_1-L_6 r_{13})^2 +(df_2-L_6 r_{23})^2+(df_3-L_6 r_{33}-L_1)-L_2^2}$
\item $q_3 = atan2(\pm\sqrt{1-a^2},a) \leftarrow a=\dfrac{x^2+y^2+(z-L_1)^2-D_3^2-D_2^2}{2D_2D_3}$
\item $q_2 = atan2(zD_3c_{q_3} + zD_2 - L_1D_3c_{q_3} - L_1D_2 \pm \sqrt{x^2+y^2}D_3s_{q_3},$

\hspace{20mm} $ \mp \sqrt{x^2+y^2}D_3s_{q_3} \mp \sqrt{x^2+y^2}D_2 + zD_3s_{q_3} - L_1D_3c_{q_3})$
\item $q_1 = atan2(\dfrac{y}{-a},\dfrac{x}{-a})$
\end{itemize}

\subsection{Calcolo Orientamento disaccoppiato}
Arrivati qui basta calcolare il risultato di $ R_{03}^T(q_a) \cdot R $ per ottenere la matrice da eguagliare a $R_{36}(q_b)$
Calcolato il risultato e' possibile risalire quindi anche ai valori degli ultimi 3 giunti



















\end{document}